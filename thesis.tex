\documentclass[14pt,a4paper,openany]{book}
%\degree{"Доктор"} 
%\thesistitle{МОДЕЛИРАНЕ И ОПТИМИЗАЦИЯ НА КОМУНИКАЦИОННИ СТРАТЕГИИ В УПРАВЛЕНИЕ НА ИНФОРМАЦИОННИ ПРОЦЕСИ}
%\author{\href{https://www.iict.bas.bg/mo/bg/mateeva.htm}{Гергана Петкова \textsc{Матеева}}}
%\supervisor{\href{https://www.iict.bas.bg/mo/bg/atanasova.htm}{доц. д-р Татяна Владимирова \textsc{Атанасова}}}
%\addresses{ИИКТ-БАН, ул. "акад. Георги Бончев", блок 2, етаж 5, кабинет 517, град София 1113, България} 
%\subject{01.01.12 "Информатика" \\ 4.6 "Информатика и Компютърни науки" \\ 4 "Природни науки, математика и информатика"} 
%\university{\href{http://bas.bg}{Българска академия на науките}}
%\faculty{\href{http://iict.bas.bg}{Институт по информационни и комуникационни технологии}} 
%\department{\href{http://iinf.bas.bg}{Моделиране и оптимизация}}

% Добавя възможност за сензитивни хипер-връзки в самия документ.
\usepackage[pdftex, bookmarks, linktocpage]{hyperref}

% Команда с множество опции за настройка на поведението на пакета hyperref, с най-полезната опция - кирилизация на заглавията от Bookmarks в Acrobat.
\hypersetup{unicode=true, colorlinks=true, linkcolor=black, citecolor=black, urlcolor=black}

% Използване на български език.
\usepackage[T2A,T1]{fontenc}
\usepackage[utf8]{inputenc}
\usepackage[english,bulgarian]{babel}

% Използва се за групиране на изображения.
\usepackage{subcaption}

% Използване на графика.
\usepackage[pdftex]{graphicx}

% Използва се за библиографията.
\usepackage{natbib} 

% Използване на по прецизни позиции за изображенията.
\usepackage{float}

% Използване на PDF-и за кориците.
\usepackage{pdfpages}

% Използване на хедър и футър.
\usepackage{fancyhdr}

% Използване на кавички при цитиране.
%\usepackage{dirtytalk}

% Използва се за създаване на азбучен указател.
\usepackage{imakeidx}

% Използва се за листинги с програмен код.
\usepackage{listings}

% Използва се за оцветяване на клетките в таблиците.
\usepackage{xcolor,colortbl}

% Използва се за многоредови коментари.
\usepackage{verbatim}

% Използвасе за междуредово разстояние.
\usepackage{lipsum}

% Използва се за таблици, които да са на повече от една страница.
\usepackage{longtable}

% Заглавие.
\title{Моделиране и оптимизация на комуникационни стратегии в управление на информационни процеси}
\makeatletter
\let\inserttitle\@title
\makeatother

% Автор.
\author{Гергана Петкова Матеева}

% Директория с изображения.
\graphicspath{{images/}}

% Избор на активен език.
\selectlanguage{bulgarian}

% Текстове за декорация на страницата в горната и долната част.
\pagestyle{fancy}
\fancyhf{}
\fancyhead[LE,RO]{\thepage}
\fancyhead[RE]{\tiny\MakeUppercase{\inserttitle}}
\fancyhead[LO]{\tiny\leftmark}
\fancyfoot[LE,RO]{Гергана Матеева - ИИКТ-БАН - София - 2024}

% Дебелина на разделителните линии.
\renewcommand{\headrulewidth}{2pt}
\renewcommand{\footrulewidth}{1pt}

% Генериране на азбучен указател.
\onecolumn
\makeindex[columns=2, title=Азбучен указател, intoc]

% Подменя думата използван а за ноемрация на фрагментите програмен код.
\renewcommand{\lstlistingname}{Листинг}

% Смяна на названието за списъка от листингите.
\renewcommand{\lstlistlistingname}{Списък на листингите}

% Определя характеристиките на листигните за програмния код.
\lstset{backgroundcolor=\color{gray!30}, breaklines=true, language=r, frame=single}

% Разстояние от ред и половина.
\linespread{1.5}

% Начало на документа.
\begin{document}

% Стил за библиографията.
\bibliographystyle{alpha}

% Предна корица.
\includepdf[pages={1}]{covers/front}
\thispagestyle{empty}

% Номериране на страниците със служебна информация.
\pagenumbering{roman}
\setcounter{page}{1}

% Таблица на съдържанието.
\addcontentsline{toc}{chapter}{Съдържание}
\tableofcontents

% Списък с абревиатурите и съкращенията.
\addcontentsline{toc}{chapter}{Списък на съкращенията}
\chapter*{Списък на съкращенията}

\begin{longtable}{ | p{3.75cm} | c | p{3.75cm} | c | }
\hline
\cellcolor{gray!15}Чуждоезичен термин & \cellcolor{gray!15}Съкращение & \cellcolor{gray!15}Български термин & \cellcolor{gray!15}Съкращение \\ [0.05ex] 
\hline
\hline
\end{longtable}

% Списък с фигурите.
\addcontentsline{toc}{chapter}{Списък на фигурите}
\listoffigures

% Списък с таблиците.
\addcontentsline{toc}{chapter}{Списък на таблиците}
\listoftables

% Списък с листингите.
\addcontentsline{toc}{chapter}{Списък на листингите}
\lstlistoflistings

\newpage

% Номериране на страниците с основното изложение.
\pagenumbering{arabic}
\setcounter{page}{1}

% Увод.
\addcontentsline{toc}{chapter}{Увод}
\chapter*{Увод}
\markboth{Увод}{}

\section*{Проблем}

\section*{Цел}

\section*{Задачи}

\section*{Структура}

Дисртационният труд е организиран от въведение, четири глави, заключение и приложения. Изложението е в ??? страници, ?? фигури, ?? таблици, ?? листинги и ??? литературни източника в библиографията. По дисертационния труд има ?? публикации, като ?? от тях са доклади на международни конференции, а ?? са публикувани в национални издания с международна видимост. 

В първа глава е ...

Във втора глава е ...

В трета глава е ...

В четвърта глава е ...

В заключението е направено обобщение на извършените изследвания, приложен е списък с публикациите по дисертационния труд, представен е списък със забелязани цитирания, дадени са насоки за бъдещо развитие, направено е обобщение на получените резултати.


% Отделните глави са в отделни файлове.
\chapter{Избор на банери за визуализация}

\section{}

Подход за съставяне на график за визуализация на банери, с помощта на Лагранж декомпозиция е представен в \cite{10.1145/945846.945848}. Определяне на възможно най-ефективно местоположение на банерите е адресирано в \cite{Kaul2018}.

\section{}

Съставяне на график за визуализиране на банери с помощта на невронни мрежи и генетични алгоритми е представен в \cite{DEANE20125168}. Целево програмиране в комбинация с генетични алгоритми е използвано за персонализиране на банер рекламни съобщения в \cite{KARUGA200185}. Хибриден алгоритъм, като комбинация между най-големия размер запълва най-много (Largest Size Most Full) и генетичен алгоритъм е предложен в \cite{KUMAR20061067}.

\section{}

\section{Дискусия и изводи}

\chapter{}

\section{}

\section{}

\section{}

\section{Обобщение}

\include{chapters/chapter03}
\include{chapters/chapter04}

% Заключение.
\addcontentsline{toc}{chapter}{Заключение}
\chapter*{Заключение}
\markboth{Заключение}{}

\section*{Резюме на получените резултати}

В дисертационния труд са постигнати резултати с научно-приложен и приложен характер, както следва:

\subsection*{Научно-приложни резултати}

\begin{description}
\item 1. ;
\item 2. ;
\item 3. ;
\end{description}

\subsection*{Приложни резултати}

\begin{description}
\item 4. ;
\item 5. ;
\item 6. ;
\end{description}

\section*{Насоки за бъдещи изследвания}

\section*{Публикации по темата на дисертационния труд}

\begin{itemize}
\item
\item
\item
\end{itemize}

\section*{Забелязани цитирания}

\begin{itemize}
\item 
	\begin{itemize}
	\item
	\item
	\item
	\end{itemize}

\item 
	\begin{itemize}
	\item
	\item
	\item
	\end{itemize}

\item 
	\begin{itemize}
	\item
	\item
	\item
	\end{itemize}
\end{itemize}

\newpage
\section*{Декларация за оригиналност на резултатите}

\vspace{1cm}

Декларирам, че настоящата дисертация съдържа оригинални резултати, получени при проведени от мен научни изследвания, с подкрепата и съдействието на научния ми ръководител. Резултатите, които са получени, описани и/или публикувани от други учени, са надлежно и подробно цитирани в библиографията.

Настоящата дисертация не е прилагана за придобиване на научна степен в друго висше училище, университет или научен институт.

\vspace{2cm}

\begin{tabular}{ c c c c }
Дата: & .......................................... & Подпис: & .......................................... \\ 
& гр. София & & / Гергана Матеева / \\  
\end{tabular}


% Списък с използвана литература и източници на информация.
\addcontentsline{toc}{chapter}{Библиография}
%\chapter*{Библиография}

\bibliography{chapters/references}


% Азбучен указател на използваните термини.
\printindex

% Приложения.
\addcontentsline{toc}{chapter}{Приложение А - програмен код}
\chapter*{Приложение А - програмен код}
\markboth{Приложение А - програмен код}{}



% Задна корица.
%\includepdf[pages=-]{covers/back}

\end{document}
