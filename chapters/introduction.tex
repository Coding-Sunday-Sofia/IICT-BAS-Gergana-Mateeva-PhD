\addcontentsline{toc}{chapter}{Увод}
\chapter*{Увод}
\markboth{Увод}{}

Класическата съвременна търговия, особено при вида биснес-към-клиент (B2C) се подчинява на четирите принципа (4Ps) от меркетинговия микс – product, price, place, promotion. В основата на съвременната търговия е продуктът (или услугата). Второто много важно нещо е цената, тъй като тя е подвластна на баланса между търсенето, предлагането и достигането на равновесна цена. На трето място, ключов фактор е локацията за продажба. Не всяко местоположение е подходящо за всеки продукт/услуга. Последният компонент в маркетинговия микс е промотирането. Основен компонент в промотирането се явява рекламата \cite{POCOL170134}. 

Настоящият дисертационен труд разглежда рекламните възможности за осъществяване на промоции в Интернет. 

\section*{Проблем}

Относително голям дял от Интернет рекламата е под формата на банери \cite{1597398, usmonova_dilfuza_ilkhomovna_usmanov_shak_2022_7110290}, от където идват голяма част от приходите на много уеб сайтове \cite{7325200}. Основен проблем е как максимално точно да бъдат напаснати зрителите на рекламни съобщения \cite{10.1145/2939672.2939724}, спрямо предлаганото от рекламодателите \cite{MIRALLESPECHUAN201839}. Рекламата е ефективна, когато правилното рекламно послание попадне при правилния зрител \cite{10.1145/2339530.2339655}. Милионите потребители в Интернет имат твърде разнородни интереси, спрямо стотиците хиляди рекламодатели \cite{10.1145/3097983.3098134}. Всичките възможности за напасване правят проблема NP-труден \cite{KIM2020106226}.

\section*{Цел}

Целта, поставена в настоящия дисертационен труд, е подобряване на показателите за откликване от страна на потребителите, при визуализация на Интернет банер рекламни съобщения. 

\section*{Задачи}

За постигане на целта, поставена в настоящия дисертационен труд, се поставят за решаване набор от задачи. Част от задачите са с теоретичен характер и покриват подобряване на алгоритми за случайно разпределяне на рекламните съобщения в уеб базирана система за ротация на банери. Другата част от задачите са с приложен характер и са свързани с прилагане на подбрани оптимизационни алгоритми и практическа реализация на софтуерен модул за оптимизирана ротация на банери. Набелязаните задачи са както следва:

* Обзор и анализ на съществуващите алгоритми за избор на рекламни банери, които да се визуализират на крайните потребители;

* Обзор и анализ на оптимизационни алгоритми, които могат да подобрят процеса по напасване, между интересите на потребителите и предлаганите рекламни послания;

* Програмна реализация на софтуерен модул за оптимизация на последователността за ротация на банери;

* Извършване на сравнителен анализ за ефективността от предложената оптимизация, спрямо предходно използваните алгоритми за ротация на банери;

\section*{Структура}

Дисртационният труд е организиран от въведение, четири глави, заключение и приложения. Изложението е в ??? страници, ?? фигури, ?? таблици, ?? листинги и ??? литературни източника в библиографията. По дисертационния труд има ?? публикации, като ?? от тях са доклади на международни конференции, а ?? са публикувани в национални издания с международна видимост. 

В първа глава е ...

Във втора глава е ...

В трета глава е ...

В четвърта глава е ...

В заключението е направено обобщение на извършените изследвания, приложен е списък с публикациите по дисертационния труд, представен е списък със забелязани цитирания, дадени са насоки за бъдещо развитие, направено е обобщение на получените резултати.

