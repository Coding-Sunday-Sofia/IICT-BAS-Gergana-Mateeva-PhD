\chapter{Избор на банери за визуализация}

Банер реклама е една от най-разпространените форми на онлайн реклама от самото начало на Интернет. Банер рекламите могат да бъдат намерени на почти всеки уеб сайт и са се доказали като ефективен начин за популяризиране на продукти и услуги. Един от най-популярните начини за показване на рекламни банери е чрез програми за обмен на банери. Тези програми позволяват на собствениците на уеб сайтове да показват реклами на уеб сайтовете на другия в замяна на показване на свои собствени реклами. В това изложение ще се разгледа оптимизацията на програмите за обмен на Интернет банери.

Програмите за обмен на банери работят на прост принцип - вие показвате реклама на уеб сайта си и в замяна вашата реклама се показва на уеб сайт на някой друг. Идеята е да увеличите показването на вашите реклами и да привлечете трафик към уеб сайта си. Програмите за обмен на банери съществуват от дълго време и са се развили, за да станат по-сложни през годините.

Оптимизирането на програмите за обмен на банери включва максимално увеличаване на показването на вашите реклами, като същевременно минимизира броя на импресиите, които трябва да направите, за да получите тези показвания. С други думи, искате да сте сигурни, че рекламите ви се показват на уеб сайтове, които е най-вероятно да генерират кликвания, потенциални клиенти или продажби за вашия бизнес.

Един от най-важните фактори, които трябва да имате предвид, когато оптимизирате вашата програма за обмен на банери, е уместността на уеб сайтовете, показващи вашите реклами. Важно е да показвате рекламите си на уеб сайтове, които са свързани с вашата ниша или индустрия. Например, ако продавате спортно оборудване, искате рекламите ви да се показват на уеб сайтове, които обслужват любителите на спорта. Това ще гарантира, че вашите реклами се виждат от хора, които се интересуват от вашите продукти и услуги.

Друг важен фактор, който трябва да имате предвид, е местоположението на вашите реклами на уеб сайтовете, които ги показват. Разположението на вашите реклами може значително да повлияе на тяхната ефективност. В идеалния случай искате вашите реклами да бъдат поставени в зони с голяма видимост, като например в горната част на страницата или близо до съдържанието. Това ще гарантира, че вашите реклами ще бъдат видени от максимален брой посетители на уеб сайта.

Дизайнът на вашите реклами също е важен фактор за оптимизирането на вашата програма за обмен на банери. Вашите реклами трябва да бъдат визуално привлекателни и да съдържат ясен призив за действие. Те също трябва да са в съответствие с имиджа и посланието на вашата марка. Ако вашите реклами са лошо проектирани или не съответстват на вашата марка, е малко вероятно те да генерират кликвания или потенциални клиенти.

И накрая, трябва редовно да наблюдавате и анализирате ефективността на вашата програма за обмен на банери. Това ще ви позволи да идентифицирате области, в които можете да подобрите и оптимизирате рекламите си допълнително. Можете да използвате инструменти като Google Analytics, за да проследите ефективността на вашите реклами и да определите кои уеб сайтове генерират най-много трафик към вашия сайт.

Програмите за обмен на банери могат да бъдат рентабилен начин за рекламиране на вашия бизнес онлайн. Въпреки това, за да увеличите максимално ефективността им, трябва да оптимизирате рекламите си и да се уверите, че се показват на подходящи уеб сайтове в зони с голяма видимост. Като наблюдавате и анализирате ефективността на вашите реклами редовно, можете да идентифицирате области за подобрение и да оптимизирате рекламите си допълнително. С правилния подход програмите за обмен на банери могат да ви помогнат да достигнете до нови аудитории и да развиете бизнеса си онлайн.

Ключов компонент в подобряването на ефективността от банер рекламите е събирането на достатъчно информативни данни \cite{LANGHEINRICH19991259}.

\section{Точни числени методи}

Точните числени методи могат да бъдат приложени при оптимизирането на обмена на банери в Интернет, за да се оптимизира разпределението на рекламни банери към уеб сайтове въз основа на набор от ограничения и цели. Тези методи включват използване на математически алгоритми за намиране на оптималното решение, което удовлетворява дадените ограничения и максимизира определената целева функция.

Ето общ преглед на това как точните числени методи могат да бъдат приложени в оптимизацията за обмен на Интернет банери:

1. Формулиране на оптимизационната задача: Първата стъпка е да се формулира задачата за оптимизация. Това включва дефиниране на променливите за вземане на решение (например - броя пъти показване на рекламен банер на уеб сайт), ограниченията (например - максималния брой пъти, на които може да се показва рекламен банер) и целевата функция (например - максимизиране на честота на кликване).

2. Разработване на модел: След като задачата е формулирана, се разработва математически модел с помощта на техники като линейно програмиране, целочислено програмиране или динамично програмиране. Моделът е предназначен да оптимизира разпределението на банер рекламите към уеб сайтове, като същевременно удовлетворява дадените ограничения и максимизира целевата функция.

3. Събиране на данни: Данните, необходими за модела за оптимизация, включват информация за банер рекламите (например - размер, формат, съдържание), уеб сайтовете, където ще се показват (например - обем на трафика, демографски данни на аудиторията) и историческото представяне на рекламите (например - честота на кликване).

4. Търсене на решение: Оптимизационният модел се решава с помощта на точен числен метод, като симплекс метода, метода на разклоняване и свързване или алгоритъм за динамично програмиране. Алгоритъмът търси оптималното решение чрез итеративно оценяване на различни комбинации от променливи на решение, докато намери решението, което удовлетворява ограниченията и максимизира целевата функция.

5. Внедряване на решение: След като бъде намерено оптималното решение, то се прилага чрез разпределяне на рекламните банери към подходящите уеб сайтове въз основа на решението. Процесът на внедряване включва програмиране на рекламния сървър да доставя рекламите на избраните уеб сайтове в съответствие с оптималното разпределение.

6. Мониторинг на модела: Ефективността на модела за оптимизация се наблюдава непрекъснато, за да се гарантира, че той остава ефективен и точен. С появата на нови данни може да се наложи моделът да бъде актуализиран или преквалифициран, за да се подобри неговата производителност.

Точните числени методи могат да бъдат ефективен инструмент за оптимизиране на разпределението на рекламни банери при оптимизиране на обмена на банери в Интернет. Чрез използване на математически алгоритми за намиране на оптималното решение, което удовлетворява дадените ограничения и максимизира целевата функция, рекламодателите могат да увеличат максимално възвръщаемостта на инвестициите си, като показват своите реклами на най-подходящата аудитория.

Подход за съставяне на график за визуализация на банери, с помощта на Лагранж декомпозиция е представен в \cite{10.1145/945846.945848}. Определяне на възможно най-ефективно местоположение на банерите е адресирано в \cite{Kaul2018}. Показването на оптимално подмножество на банер реклами е разгледано в \cite{https://doi.org/10.1002/jos.74}. Оптимизация чрез разлагане на Лагранж е предложена в \cite{doi:10.1287/ijoc.1020.0003}.

\section{Генетични алгоритми и изкуствени невронни мрежи}

\subsection{Генетични алгоритми}

Генетичните алгоритми са вид алгоритъм за оптимизация, който използва принципите на естествения подбор и генетиката, за да намери оптимални решения на даден проблем. Една област, в която са приложени генетичните алгоритми, е оптимизирането на обмена на Интернет банери.

Интернет обменът на банери е вид реклама, при която уеб сайтовете се съгласяват да показват банер реклами на другия в замяна на показване на свои собствени реклами. Целта на оптимизирането на обмена на банери е да се увеличи максимално броят кликвания или реализации, генерирани от дадена банер реклама, като се постави на най-подходящите уеб сайтове.

Ето стъпките за прилагане на генетични алгоритми в оптимизацията за обмен на Интернет банери:

1. Дефинирайте оптимизационната задача: Първата стъпка е да дефинирате оптимизационната задача, като посочите целите и ограниченията на оптимизацията. Например, целта може да е да се увеличи максимално броят кликвания, генерирани от даден банер, докато ограниченията могат да включват ограничен бюджет или ограничен брой налични уеб сайтове за показване на рекламата.

2. Дефинирайте жизнената функция: Жизнената функция е мярка за това колко добре се представя дадено решение (в този случай набор от уеб сайтове за показване на рекламния банер) при постигане на целите за оптимизация. Например, фитнес функцията може да изчисли броя на кликванията, генерирани от рекламния банер, когато се показва на определен набор от уеб сайтове.

3. Генериране на първоначална популация: Следващата стъпка е да се генерира първоначална популация от кандидат-решения. В този случай кандидат-решенията са набори от уеб сайтове за показване на рекламния банер.

4. Оценете жизнеността на всяко решение: Жизнената функция се използва за оценка на пригодността на всяко решение в популацията.

5. Изберете най-подходящите решения: Най-подходящите решения се избират, за да формират основата на следващото поколение решения.

6. Генерирайте нови решения чрез кръстосване и мутация: Новите решения се генерират чрез комбиниране на елементи от най-подходящите решения чрез кръстосване и чрез произволно мутиране на елементи от решенията.

7. Оценете жизнеността на новите решения: Функцията за пригодност се използва за оценка на пригодността на новите решения.

8. Повторете стъпки 5-7 до прага за сходимост: Стъпки 5-7 се повтарят, докато не бъде изпълнен критерий за сближаване (например - достигнат максимален брой повторения или пригодността на решенията спре да се подобрява).

Генетичните алгоритми могат да бъдат приложени за оптимизиране на обмена на Интернет банери чрез генериране на популация от кандидат-решения (набори от уеб сайтове за показване на рекламния банер), оценка на тяхната годност (брой генерирани кликвания) и генериране на нови решения чрез кръстосване и мутация. Най-подходящите решения се избират, за да формират основата на следващото поколение решения, и процесът се повтаря, докато не бъде изпълнен критерий за конвергенция.

\subsection{Изкуствени невронни мрежи}

Изкуствените невронни мрежи могат да се използват при оптимизиране на обмена на банери в Интернет за подобряване на насочването и ефективността на банер рекламата. Банера реклама е популярна форма на онлайн реклама, която включва показване на банер реклами на уеб сайтове с цел популяризиране на продукти или услуги на посетителите на уеб сайта.

Целта на оптимизирането на обмена на банери е да се увеличи честотата на кликване (CTR) на банер рекламите, което е процентът на хората, които кликват върху рекламата, след като я видят. Един от начините да постигнете това е чрез насочване към правилната аудитория, където изкуствените невронни мрежи могат да бъдат полезни.

Ето общ преглед на това как изкуствените невронни мрежи могат да се прилагат при оптимизиране на обмена на банери в Интернет:

1. Събиране на данни: Първата стъпка е да съберете данни за рекламните банери и уеб сайтовете, където се показват. Тези данни могат да включват информация за самата реклама (например - размер, формат, съдържание), както и информация за уеб сайта (например - обем на трафика, демографски данни на аудиторията и други).

2. Подготовка на данните: Събраните данни трябва да бъдат подготвени за използване в изкуствената невронна мрежа. Това включва почистване на данните, премахване на липсваща или неподходяща информация и трансформиране на данните във формат, който може да се използва от мрежата.

3. Разработване на модел: Моделът на изкуствената неверонна се разработва с помощта на подготвените данни. Моделът се обучава с помощта на набор от входни данни (например - характеристики на уеб сайта, потребителско поведение) и изходни данни (например – честота на кликане). След това моделът се настройва фино с помощта на набор за валидиране, за да се гарантира, че работи добре при нови данни.

4. Внедряване на модел: След като моделът бъде обучен и валидиран, той може да бъде внедрен за оптимизиране на обмена на банери. Изкуствената невронна мрежа използва входните данни (например - характеристики на уеб сайта, потребителско поведение), за да предскаже вероятността даден потребител да щракне върху определен рекламен банер. След това рекламата с най-висока прогнозиран честота за кликане се избира за показване на уеб сайта.

5. Мониторинг на модела: Ефективността на изкуствената неверонна мржа се наблюдава непрекъснато, за да се гарантира, че тя все още е точна и ефективна. С появата на нови данни може да се наложи моделът да бъде преобучен или актуализиран, за да се подобри неговата производителност.

Изкуствените невронни мрежи могат да бъдат мощен инструмент за подобряване на насочването и ефективността на банер рекламата при оптимизирането на обмена на банери в Интернет. Чрез използване на изкуствени невронни мрежи за прогнозиране на вероятността потребител да щракне върху конкретна банер реклама, рекламодателите могат да увеличат максимално възвръщаемостта на инвестициите си, като показват най-ефективните реклами на най-подходящата аудитория.

\subsection{Реализации}

Съставяне на график за визуализиране на банери с помощта на невронни мрежи и генетични алгоритми е представен в \cite{DEANE20125168}. Комбинация от генетичен алгоритъм и Калман филтър, за съставяне на график за визуализацията на реклами е предложен в \cite{doi:10.1287/mksc.17.3.214}. Целево програмиране в комбинация с генетични алгоритми е използвано за персонализиране на банер рекламни съобщения в \cite{KARUGA200185}. Хибриден алгоритъм, като комбинация между най-големия размер запълва най-много (Largest Size Most Full) и генетичен алгоритъм е предложен в \cite{KUMAR20061067}. Точните числени методи за оптимално разполагане на рекламни банери в уеб страници са развити до хибридна реализация с генетични алгоритми в \cite{kumar2001hybrid}.

\section{Експертни системи}

Експертните системи могат да се прилагат при оптимизиране на обмена на банери в Интернет за подобряване на насочването и ефективността на банер рекламата. Експертните системи са компютърни програми, които имитират способностите за вземане на решения на човешки експерт в определена област, като кодират експертните знания в набор от правила или евристики.

Ето общ преглед на това как експертните системи могат да се прилагат при оптимизиране на обмена на Интернет банери:

1. Придобиване на знания: Първата стъпка е да придобиете знанията на експертите в областта. Това знание може да включва информация за банер рекламите (например - размер, формат, съдържание), уеб сайтовете, където те ще бъдат показани (например - обем на трафика, демографски данни на аудиторията) и историческата ефективност на рекламите (например – честота на кликване). Тези знания могат да бъдат придобити чрез интервюта, проучвания или други техники за събиране на данни.

2. Разработване на база от правила: След това придобитите знания се кодират в набор от правила или евристики, които определят процеса на вземане на решения за оптимизиране на обмена на банери. Базата от правила е разработена от експерти по области или инженери по знания, които използват своя опит, за да преведат придобитите знания в набор от правила, които могат да бъдат използвани от експертната система.

3. Разработване на модел: Експертната система е разработена с помощта на базата от правила и други софтуерни инструменти, като език за представяне на знания или базиран на правила език за програмиране. Системата е проектирана да взема решения кои рекламни банери да се показват на кои уеб сайтове въз основа на въведените данни, като характеристиките на рекламата, уеб сайта и аудиторията.

4. Внедряване на модел: След като експертната система бъде разработена и валидирана, тя може да бъде внедрена за оптимизиране на обмена на банери. Системата взема входните данни, като характеристиките на рекламата, уеб сайта и аудиторията, и прилага базата от правила, за да вземе решения кои рекламни банери да се показват на кои уеб сайтове.

5. Мониторинг на модела: Ефективността на експертната система се наблюдава непрекъснато, за да се гарантира, че е точна и ефективна. Когато нови данни станат достъпни, може да се наложи системата да бъде актуализирана или преквалифицирана, за да се подобри нейната производителност.

Експертните системи могат да бъдат мощен инструмент за подобряване на насочването и ефективността на банер рекламата при оптимизиране на обмена на банери в Интернет. Чрез кодиране на експертизата на експертите по области в набор от правила или евристики, системата може да взема интелигентни решения за това кои рекламни банери да показва на кои уеб сайтове, увеличавайки максимално възвръщаемостта на инвестициите за рекламодателите.

Използването на експертни системи при обмена на банер реклами е предложен в \cite{Krasteleva_Soshnikov_2002}. Рекламирането, основаващо се на данни, става все по популярно през последните години \cite{Stange2014}. Същината е, че решенията за визуализиране на рекламни съобщения се взема в реално време.

\section{Дискусия и изводи}

От направеното проучване, ясно се откроява нуждата за предиктивно избиране на рекламните съобщения за визуализация \cite{doi:10.1089/big.2015.0006}. 

