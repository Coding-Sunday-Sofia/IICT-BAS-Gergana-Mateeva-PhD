\chapter{Избор на банери за визуализация}

\section{Точни числени методи}

Подход за съставяне на график за визуализация на банери, с помощта на Лагранж декомпозиция е представен в \cite{10.1145/945846.945848}. Определяне на възможно най-ефективно местоположение на банерите е адресирано в \cite{Kaul2018}.

\section{Изкуствени невронни мрежи и генетични алгоритми}

Съставяне на график за визуализиране на банери с помощта на невронни мрежи и генетични алгоритми е представен в \cite{DEANE20125168}. Комбинация от генетичен алгоритъм и Калман филтър, за съставяне на график за визуализацията на реклами е предложен в \cite{doi:10.1287/mksc.17.3.214}. Целево програмиране в комбинация с генетични алгоритми е използвано за персонализиране на банер рекламни съобщения в \cite{KARUGA200185}. Хибриден алгоритъм, като комбинация между най-големия размер запълва най-много (Largest Size Most Full) и генетичен алгоритъм е предложен в \cite{KUMAR20061067}.

\section{Експертни системи}

Използването на експертни системи при обмена на банер реклами е предложен в \cite{Krasteleva_Soshnikov_2002}. Рекламирането, основаващо се на данни, става все по популярно през последните години \cite{Stange2014}. Същината е, че решенията за визуализиране на рекламни съобщения се взема в реално време.

\section{Дискусия и изводи}

От направеното проучване, ясно се откроява нуждата за предиктивно избиране на рекламните съобщения за визуализация \cite{doi:10.1089/big.2015.0006}. 

